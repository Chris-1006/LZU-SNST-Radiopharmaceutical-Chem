% ----------------------------------------------------- 
% 引入所需的宏包
\usepackage{tikz} % 用于绘图
\usetikzlibrary{calc} % 计算库
\usepackage{eso-pic} % 用于添加背景图片
\AddToShipoutPictureBG{%
  \begin{tikzpicture}[overlay,remember picture]
    \draw[line width=0.6pt] % 设置边框线的粗细
    ($ (current page.north west) + (0.6cm,-0.6cm) $) % 边框左上角位置
    rectangle
    ($ (current page.south east) + (-0.6cm,0.6cm) $); % 边框右下角位置
  \end{tikzpicture}}

% 设置章节编号深度
\setcounter{secnumdepth}{2} % 包含 subsection 编号

% 修改 subsection 编号格式为 1.1, 1.2 等
\renewcommand{\thesubsection}{\thesection.\arabic{subsection}}

% 定义颜色
\usepackage{xcolor}
\definecolor{c1}{HTML}{4F80BC} % 目录颜色
\definecolor{c2}{RGB}{190,20,83} % 引用颜色

% 设置中文支持
\usepackage{ctex}

% 设置页面布局
\usepackage[a4paper,top=28mm,bottom=28mm,left=15mm,right=15mm]{geometry}

% 引入超链接包
\usepackage{hyperref}
\hypersetup{
  colorlinks, % 设置链接颜色
  linktoc = section, % 超链接位置,选项有 section, page, all
  linkcolor = c1, % 链接颜色
  citecolor = c1 % 引用颜色
}

% 引入其他常用宏包
\usepackage{amsmath,enumerate,multirow,float} % 数学、列表和表格
\usepackage{tabularx} % 可变宽度表格
\usepackage{tabu} % 扩展表格功能
\usepackage{subfig} % 支持子图
\usepackage{fancyhdr} % 自定义页眉页脚
\usepackage{graphicx} % 图形处理
\usepackage{wrapfig} % 图片环绕文本
\usepackage{physics} % 物理符号
\usepackage{appendix} % 附录处理
\usepackage{amsfonts} % 数学字体
\usepackage{lastpage} % 引入lastpage宏包,用于获取总页数

% 自定义彩色盒子
\usepackage{tcolorbox}
\tcbuselibrary{skins,breakable} % 加载skins和breakable库
\newtcolorbox{tbox}[2][]{
  colframe=black!70!, % 边框颜色
  breakable, % 允许跨页
  enhanced, % 增强功能
  boxrule =0.5pt, % 边框线宽
  title = {#2}, % 盒子标题
  fonttitle = \large\kaishu\bfseries, % 标题字体
  drop fuzzy shadow, % 阴影效果
  #1 % 额外选项
}

% 页眉页脚设置
\fancypagestyle{plain}{\pagestyle{fancy}} % 设置页面样式
\pagestyle{fancy} % 应用自定义样式
\fancyhf{} % 清空当前的页眉和页脚
\fancyhead[C]{\textbf{\nouppercase{\leftmark}}} % 中间页眉显示章节名称(无大写)

% 设置页脚为当前页数/总页数格式
\fancyfoot[C]{\thepage\ / \pageref{LastPage}} % 页脚,中间显示 "当前页数/总页数"

% 保留章节编号
% \renewcommand{\thesection}{} % 不再去掉章节编号

% 设置目录名称
\renewcommand{\contentsname}{\centerline{\huge 目录}} % 目录标题居中显示

% 设置章节标题格式
\usepackage{titlesec}
\usepackage{titletoc}
\titleformat{\section}{\centering\Large}{}{1em}{} % 设置章节标题格式

% listing代码环境设置
\usepackage{listings} % 引入代码高亮包
\lstloadlanguages{python} % 加载Python语言
\lstdefinestyle{pythonstyle}{
  backgroundcolor=\color{gray!5}, % 背景颜色
  language=python, % 代码语言
  frameround=tftt, % 圆角框
  frame=shadowbox, % 框类型
  keepspaces=true, % 保持空格
  breaklines, % 自动换行
  columns=spaceflexible, % 列宽自适应
  basicstyle=\ttfamily\small, % 基本文本设置,字体为teletype,大小为小号
  keywordstyle=[1]\color{c1}\bfseries, % 关键词样式
  keywordstyle=[2]\color{Red!70!black}, % 关键词样式
  stringstyle=\color{Purple}, % 字符串样式
  showstringspaces=false, % 不显示字符串中的空格
  commentstyle=\ttfamily\scriptsize\color{green!40!black}, % 注释样式
  tabsize=2, % 制表符宽度
  morekeywords={as}, % 更多关键词
  morekeywords=[2]{np, plt, sp}, % 更多关键词
  numbers=left, % 代码行数显示在左边
  numberstyle=\it\tiny\color{gray}, % 行号样式
  stepnumber=1, % 每行显示行号
  rulesepcolor=\color{gray!30!white} % 规则分隔颜色
}

% 定义度数符号
\def\degree{${}^{\circ}$}

% 设置图片路径
\graphicspath{{./images/}} % 图片相对路径
\allowdisplaybreaks[4] % 允许公式跨页
